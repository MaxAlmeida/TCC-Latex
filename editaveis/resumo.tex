\begin{resumo}
 O uso de ambientes virtualizados vem tendo um aumento crescente nos últimos anos e isso
se deve em parte ao crescente uso da computação em nuvem. Desse modo muitos serviços de infraestrutura vem sendo disponibilizados via internet. Possibilitando assim, que em um mesmo servidor sejam executados isoladamente diversos sistemas operacionais. Esse compartilhamento de recursos é um dos benefícios que a virtualização provê, o que, consequentemente reduz os custos com recursos físicos (servidores) e possibilita facilidades no que diz respeito a organização e gerenciamento da infraestrutura a partir de mecanismos de escalonamento de recursos, por exemplo. Outros benefícios que propiciam o uso de ambiente virtualizados incluem segurança, alta disponibilidade e tolerância a falha. Entretanto, a virtualização traz consigo desafios no que diz respeito predição e gerenciamento de desempenho de sistemas virtualizados. Desse modo, aplicações disponibilizadas em ambientes virtualizados podem ter uma diferença considerável de desempenho com relação as mesmas aplicações disponibilizadas em máquinas físicas convencionais. Um dos fatores que colaboram para diferença de desempenho são as possíveis interferências aos quais máquinas virtuais são submetidas quando, na execução de seus processos , estão sendo disponibilizadas em um mesmo servidor. Desta forma, outros questionamentos podem ser feitos, com relaçao a influência  do tipo de virtualizaçao e de monitores de maquinas virtuais, na interferencia de desempenho. Com isso, a partir de trabalhos já realizados, esse trabalho propõe a aplicação de um estudo de interferência de desempenho em ambiente virtuais levando se em conta o tipo de virtualização provida pelo monitor de máquinas virtuais. 




%Outro fator que deve-se levar em conta com relação ao impacto no desempenho do sistema é a alocação de CPU para uma determinada máquina virtual. A partir disso, entende-se que é altamente desejável identificar os fatores que levam aos custos da virtualização bem como quantifica-los. Neste contexto, levando -se em conta as diversas tecnologias existentes de virtualização, este trabalho propõe uma metodologia genérica para analisar a influencia da virtualização na performance de ambientes virtualizados. Pretende-se, também, a implementação dessa metodologia em uma nuvem privada como estudo de caso.

 \vspace{\onelineskip}
    
 \noindent
 \textbf{Palavras-chaves}: Computação em Nuvem. Virtualização. Performance de Sistemas. KVM.
\end{resumo}
