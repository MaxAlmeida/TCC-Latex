\begin{resumo}
 O uso de ambientes virtualizados vem tendo um aumento crescente nos últimos anos e isso
se deve em parte ao crescente uso da computação em nuvem. Desse modo muitos serviços de infraestrutura vem sendo disponibilizados via internet. Possibilitando assim, que em um mesmo servidor sejam executados isoladamente diversos sistemas operacionais. Esse compartilhamento de recursos é um dos benefícios que a virtualização provê, o que, consequetemente reduz os custos com recursos físicos (servidores) e possibilita facilidades no que diz respeito a organização e gerenciamento da infraestrutura a partir de mecanismos de escalonamento de recursos, por exemplo.Outros benefícios que propiciam o uso de ambiente virtualizados incluem segurança, alta disponibilidade e tolerância a falha. Entretanto, a virtualização traz consigo desafios no que diz respeito predição e gerencimaneto de desempenho de sistemas virtualizados. Desse modo, aplicações disponibilizadas em ambientes virtualizados podem ter uma diferença considerável de desempemho com relação as mesmas aplicações disponibilizadas em máquinas físicas convencionais. Outro fator que deve-se levar em conta com relação ao impacto no desempenho do sistema é a alocação de CPU para uma determinada máquina virtual. A partir disso, entende-se que é altamente desejável identificar os fatores que levam aos custos da virtualização bem como quantifica-los. Neste contexto, levando -se em conta as diversas tecnologias existentes de virtualização, este trabalho proponhe uma metodologia genérica para analisar a influencia da virtualização na performance de ambientes virtualizados. Pretende-se, também, a implementação dessa metodologia em uma nuvem privada como estudo de caso.

 \vspace{\onelineskip}
    
 \noindent
 \textbf{Palavras-chaves}: Computação em Nuvem. Virtualização. Performance. KVM.
\end{resumo}
