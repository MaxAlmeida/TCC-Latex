\begin{resumo}
 O uso de ambientes virtualizados vem tendo um aumento crescente nos últimos anos e isso se deve em parte ao crescente uso da computação em nuvem. Entretanto, aplicações disponibilizadas em ambientes virtualizados podem ter uma perda considerável de desempenho, se comparadas as mesmas aplicações disponibilizadas em máquinas físicas convencionais. Um dos fatores que colaboram para perda de desempenho, é a interferência sofrida pelas máquinas virtuais quando executadas em um mesmo servidor. Os resultados deste trabalho comprovam que além disso, essa interferência é variável dado o tipo de aplicação que está sendo executada nas máquinas virtuais. Com isso, a partir de trabalhos já realizados, esse trabalho propõe a aplicação de um estudo de interferência de desempenho em máquinas virtuais, utilizando o \textit{kvm} como \textit{hypervisor}. Para coleta de dados, foram desenvolvidos procedimentos, bem como, foram definidas ferramentas tipicamentes utilizadas para medição de desempenho em sistemas computacionais.  %Esse compartilhamento de recursos é um dos benefícios que a virtualização provê, o que, consequentemente reduz os custos com recursos físicos (servidores) e possibilita facilidades no que diz respeito a organização e gerenciamento da infraestrutura a partir de mecanismos de escalonamento de recursos. Outros benefícios que propiciam o uso de ambiente virtualizados incluem segurança, alta disponibilidade e tolerância a falha. 

%Mesmo com todas as vantagens que a virtualização pode prover, um dos desafios reside compreensão da diferença de desempenho com relação as mesmas aplicações disponibilizadas em máquinas física convencionais

%Desse modo, muitos serviços de infraestrutura vem sendo disponibilizados via internet. Possibilitando assim, que em um mesmo servidor sejam executados isoladamente diversos sistemas operacionais

%Outro fator que deve-se levar em conta com relação ao impacto no desempenho do sistema é a alocação de CPU para uma determinada máquina virtual. A partir disso, entende-se que é altamente desejável identificar os fatores que levam aos custos da virtualização bem como quantifica-los. Neste contexto, levando -se em conta as diversas tecnologias existentes de virtualização, este trabalho propõe uma metodologia genérica para analisar a influencia da virtualização na performance de ambientes virtualizados. Pretende-se, também, a implementação dessa metodologia em uma nuvem privada como estudo de caso.

 \vspace{\onelineskip}
    
 \noindent
 \textbf{Palavras-chaves}: Computação em Nuvem; Virtualização; Desempenho de Sistemas; KVM.
\end{resumo}
