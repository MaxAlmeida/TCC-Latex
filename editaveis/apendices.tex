\begin{apendicesenv}

\partapendices

\chapter{Código fonte da receita do \textit{OpenNebula Frontend}}

\begin{lstlisting}
repo_dir = '/etc/yum.repos.d/opennebula.repo'
config_ssh_dir = '/var/lib/one/.ssh/config'
package "epel-release"

template repo_dir do 
  source 'opennebula.repo.erb'
end

package 'opennebula-server'
package 'opennebula-sunstone'

execute 'disable-selinux' do
  command 'sed -i "s/SELINUX=\w*/SELINUX=disable /g" etc/sysconfig/selinux'
end

execute 'change_script' do
  command 'sed -i "s/yum install/yum install -y/g" /usr/share/one/install_gems'
end

execute 'sunstone' do
  command 'echo -e 1 "\n" "\n"| /usr/share/one/install_gems'
end

execute 'external_acess' do
  command 'sed -i "s/:host: 127.0.0.1/:host: 0.0.0.0/g" /etc/one/sunstone-server.conf'
end

service 'opennebula' do
  action [:enable, :start]
end

service 'opennebula-sunstone' do
  action [:enable, :start]
end
 
template config_ssh_dir do
  source 'config.erb'
  owner 'oneadmin'
  group 'oneadmin'
  mode '0600'
end


\end{lstlisting}
\chapter{Código fonte da receita do \textit{OpenNebula Node}}
\begin{lstlisting}
repo_dir= '/etc/yum.repos.d/opennebula.repo'

#add opennebula repository
template repo_dir do
  source 'opennebula.repo.erb'
end
#install node package for OpenNebula
package "opennebula-node-kvm"

#enable and starts services
service "messagebus.service" do
  action [:enable, :start]
end

service "libvirtd.service" do
  action [:enable, :start]
end

service "nfs-server.service" do
  action [:enable, :start]
end

\end{lstlisting}

\chapter{Código fonte da receita do \textit{Redmine}}
\begin{lstlisting}
version = '3.2.0'
url_redmine = "https://www.redmine.org/releases/redmine-#{version}.tar.gz"
redmine_dir = '/opt/redmine/'
apache_conf_dir = '/etc/apache2/sites-available/master.conf'
#install dependencies
execute 'update' do
  command "apt-get update"
  ignore_failure true
  action :nothing
end
packages = %w(mysql-client libmysqlclient-dev gcc build-essential zlib1g zlib1g-dev zlibc ruby-zip libssl-dev libyaml-dev libcurl4-openssl-dev ruby gem libapache2-mod-passenger apache2-mpm-prefork apache2-dev libapr1-dev libxslt1-dev checkinstall libxml2-dev ruby-dev vim libmagickwand-dev imagemagick)


packages.each do |package_name|
  package package_name
end

directory redmine_dir

remote_file redmine_dir + 'redmine.tar.gz' do
  source url_redmine
  mode '0755'
end

execute 'extract_redmine' do
  command 'tar xzf redmine.tar.gz'
  cwd redmine_dir
end

#installing gem bundler
gem_package 'bundler'

extracted_redmine_dir = redmine_dir+'redmine-'+version

execute 'bundle_install' do
  command 'bundle install'
  cwd extracted_redmine_dir
end

execute 'generate_redmine_secret_token' do
  command 'bundle exec rake generate_secret_token'
  cwd extracted_redmine_dir
end

template extracted_redmine_dir+'/'+'config/database.yml' do
  source 'database.yml.erb'
  variables({
    redmine_passwd: node['passwd']['redmine']
  })
end

execute 'database_migration' do
  command 'RAILS_ENV=production bundle exec rake db:migrate || RAILS_ENV=production bundle exec rake redmine:load_default_data'
  cwd extracted_redmine_dir
end

execute "chown-data-www" do
  command "sudo chown -R www-data files log tmp public/plugin_assets"
  user "root"
  cwd extracted_redmine_dir
end

execute "simbolic_link" do
  command 'sudo ln -s '+extracted_redmine_dir+'/public/'+' /var/www/html/redmine'  
end 

template apache_conf_dir do
  source 'master.conf.erb'
end

execute 'disable_default_apache' do
  command 'sudo a2dissite 000-default.conf'
end

execute 'enable_master_conf' do
  command 'sudo a2ensite master.conf'
end

execute 'passenger_permission' do
  command "echo 'PassengerUser www-data' >> /etc/apache2/mods-available/passenger.conf"
end

execute 'enable_passenger' do
  command "sudo a2enmod passenger"

end
service 'apache2' do
  action :restart
end 

\end{lstlisting}

\chapter{Código fonte da receita do servidor \textit{Proxy} usando \textit{squid}}\label{apendice}
\begin{lstlisting}
# Recipe for squid server

# Config: 
squid_maximum_object_size = "600" #MB
squid_minimum_object_size = "0" #MB
squid_cache_size = "30000" #MB

# Yum update
execute "yum_update" do
  command "yum -y update"
end

# Install Squid
package 'squid'

# Configure squid.conf
squid_conf_file = "/etc/squid/squid.conf"
template squid_conf_file do
  source 'squid.conf.erb'
  variables({
    cache_size: squid_cache_size,
    maximum_object_size: squid_maximum_object_size,
    minimum_object_size: squid_minimum_object_size
  })
end

# Restart squid service
service "squid" do
  action :restart
end

# Enable iptables to listen port 3128
execute "add_squid_iptables_rule" do
  command "iptables -I INPUT -p tcp --dport 3128 -j ACCEPT && " +
          "iptables-save > /etc/sysconfig/iptables"
end

# Install iptables service
package 'iptables-services'

# Enable iptables on boot and start it
service "iptables" do
  action [:enable,:start]
end

\end{lstlisting}

\end{apendicesenv}
