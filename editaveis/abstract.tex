\begin{resumo}[Abstract]
 \begin{otherlanguage*}{english}
The use of virtualized environments has had a growing increase in recent years and this is due in part to the increasing use of cloud computing. Thus many infrastructure services has been made available via the Internet. Thus enabling it on the same server are separately run multiple operating systems. This resource sharing is one of the benefits that virtualization provides, which consequetemente reduces the costs of physical resources (servers) and provides facilities with regard to organization and infrastructure management from resource scheduling mechanisms, for example . Other benefits that favor the use of virtualized environment include security, high availability and fault tolerance. However, virtualization brings challenges regarding prediction and management performance virtualized systems. Thus, applications deployed in virtualized environments can have a considerable difference in performance regarding the same applications available in conventional physical machines. Another factor that should be taken into account with respect to the impact on system performance is the CPU allocation for a particular virtual machine. From this, it is understood that it is highly desirable to identify the factors that lead to virtualization costs and quantifies them. In this context, taking themselves in account the various virtualization technologies, this paper proposes a generic methodology to analyze the influence of virtualization performance in virtualized environments. The aim is also to implement this methodology in a private cloud as a case study.
   \vspace{\onelineskip}
 
   \noindent 
   \textbf{Key-words}: Cloud Computing. Virtualization. Performance Systems. KVM.
 \end{otherlanguage*}
\end{resumo}
