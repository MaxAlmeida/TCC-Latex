\begin{resumo}[Abstract]
 \begin{otherlanguage*}{english}
 
The use of virtualized environments has been growing in recent years and this is due partly to the increasing use of cloud computing. However, applications deployed in virtualized environments may have a considerable loss of performance, if compared to the same applications available in conventional physical machines. One of the factors that contribute to performance loss is the interference suffered by virtual machines when running on the same server. Based on related works, this work proposes the application of a performance interference study on virtual machines using the kvm as hypervisor. For data collection, procedures were developed, as well as defined using tools typically used for measuring performance in computer systems. The results of this work show that the degree interference is variable given the type of application that is running in virtual machines. In addition, with the data, it is possible to apply statistical models for performance prediction of an application. 

 %This resource sharing is one of the benefits that virtualization provides, which consequently reduces the costs of physical resources (servers) and provides facilities with regard to organization and infrastructure management from resource scheduling mechanisms, for example . Other benefits that favor the use of virtualized environment include security, high availability and fault tolerance. However, virtualization brings challenges regarding prediction and management performance virtualized systems. Therefore, applications deployed in virtualized environments can have a considerable difference in performance regarding the same applications available in conventional physical machines. One of the factors that contribute to performance difference is the interference suffered by virtual machines when running on the same server. Thus, other questions can be made with respect to influence the type of virtualization and virtual machine monitors, the performance of interference. So, from previous work, this work proposes the application of a performance study of interference in virtual environment by taking into account the type of virtualization provided by the virtual machine monitor.

%The results of this work show that in addition, this interference is variable given the type of application that is running in virtual machines. So, from previous work, this work proposes the application of a performance interference study on virtual machines using the kvm as hypervisor. For data collection, procedures have been developed, as well as, was defined using tools typically used for measuring performance in computer systems.





%Another factor that should be taken into account with respect to the impact on system performance is the CPU allocation for a particular virtual machine. From this, it is understood that it is highly desirable to identify the factors that lead to virtualization costs and quantifies them. In this context, taking themselves in account the various virtualization technologies, this paper proposes a generic methodology to analyze the influence of virtualization performance in virtualized environments. The aim is also to implement this methodology in a private cloud as a case study.
   \vspace{\onelineskip}
 
   \noindent 
   \textbf{Key-words}: Cloud Computing; Virtualization; Performance Systems; KVM.
 \end{otherlanguage*}
\end{resumo}
