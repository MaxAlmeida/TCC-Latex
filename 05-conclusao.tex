\section{Conclusão}
\label{sec:conclusao}

Nossos resultados demonstram que as tecnologias atuais de virtualização e de
computação em nuvem não garantem um isolamento efetivo entre máquinas virtuais
em um mesmo servidor físico. A tendência é que aplicações que possuem perfil
voltado para operações em disco interferem de maneira significativa no
desempenho outras aplicações  com o mesmo perfil, mesmo quando executadas em
máquinas virtuais diferentes. Aplicações com o perfil voltado para \textit{CPU}
interferem pouco contra aplicações com o perfil voltado para aplicações em
disco e vice-versa.

A partir dos dados coletados, aplicamos modelos estatísticos para a obteção da
predição de desempenho de uma aplicação a partir das métricas de desempenho em
nível de sistema. Assim como no trabalho de Koh \cite{koh2007}, a média
ponderada com o auxílio da análise por componente principal obteve resultados
melhores se comparada com a regressão linear. Dessa forma, aplicamos um modelo
regressão polinomial que apresentou ajustes mais aproximados para a predição de
desempenho de uma aplicação.

A avaliação da degradação de desempenho entre aplicações com o perfil voltado
para utilização de \textit{cpu} não foi realizada e pode ser considerada uma
limitação deste trabalho. De toda forma, entendemos que para as configurações
de \textit{hardware} do servidor utilizado neste estudo são necessárias um
número maior de máquinas virtuais executando \textit{benchmarking} de
\textit{cpu}, ao mesmo tempo, para que fosse possível observar algum tipo de
interferência e degradação em seus desempenhos.

Por fim, os procedimentos adotados neste trabalho promovem algumas
posssibilidades para trabalhos futuros, como estudo um comparativo entre os
diversos tipos de \textit{hypervisors} utilizados, levando-se em conta as
técnicas de virtualização paravirtualização e virtualização total, de modo que
se possa avaliar qual tipo de ferramenta, dada a técnica de virtualização,
trata melhor a interferência de desempenho entre as máquinas virtuais.
