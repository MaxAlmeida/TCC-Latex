\section{Cálculo da Interferência}
\label{sec:calculo-interferencia}

Para o estudo proposto por este trabalho foi definido um  ambiente de testes
que consiste no uso de máquinas virtuais com aplicações voltadas para
realização de testes de \textit{benchmark}. Tais aplicações foram definidas a
partir do trabalho de Koh \cite{koh2007}, tendo essas sido escolhidas visando o
estresse de vários aspectos de sistema e de \textit{hardware}. A fim de se ter
comodidade na criação e destruição dessas máquinas virtuais foi utilizado o
OpenNebula como ferramenta de computação em nuvem e como \textit{hypervisor}
foi utilizado o \textit{kvm}. As máquinas virtuais possuíam, como configuração,
sistema operacional \textit{Centos 7}, espaço em disco de 15GB e 1GB de memória
\textit{RAM}. Em um primeiro momento as aplicações foram instaladas e testadas
de modo que se pudesse observar quais são os comandos utilizados para
funcionamento das mesmas, em seguida foram criados \textit{snapshots} das
máquinas virtuais com o auxílio provido pelo \textit{OpenNebula}.

Entre as aplicações escolhidas estão típicas provedoras de \textit{stress}
computacional no cotidiano, tais como compilação de código fonte, compressão e
encriptação de arquivos e processamento de imagens, bem como, ferramentas
voltadas para geração de testes de \textit{benchmark} tais como
\textit{Cachebench} e \textit{AIM Benchmark suite}. São elas: Add\_double,
Bzip2, Gzip, Ccrypt, Cachebench, Cat, Grep, cp, dd, Iozone, Make e Povray.

%Add\_double (\url{sourceforge.net/projects/aimbench}), Bzip2 (\url{bzip.org}),
%Gzip (\url{gzip.org}), Ccrypt (\url{ccrypt.sourceforge.net}), Cachebench
%(\url{icl.cs.utk.edu/projects/llcbench/cachebench.html}), Cat
%(\url{linux.die.net/man/1/cat}), Grep (\url{linux.die.net/man/1/grep}), Dp
%(\url{linux.die.net/man/1/cp}), DD (\url{linux.die.net/man/1/dd}), Iozone
%(\url{iozone.org}), Make (\url{linux.die.net/man/1/make}) e Povray
%(\url{povray.org}).

Os procedimentos adotados para o cálculo da interferência seguem os propostos
no trabalho de Koh \cite{koh2007}. Desse modo, duas máquinas virtuais são
criadas em um servidor utilizando \textit{hypervisor} \textit{KVM}. Cada
máquina virtual, denominadas \textit{'dom1'} e \textit{'dom2'} respectivamente,
executa uma das aplicações de \textit{benchmarking}. 

Uma aplicação executando em \textit{dom1} é chamada de aplicação
\textit{foreground}, e a que estiver executando em \textit{dom2} é a aplicação
\textit{background}. Por questões de notação uma aplicação \textit{foreground}
executando contra uma aplicação \textit{background} é denotada como F@B. Um dos
procedimentos adotados é garantir que a aplicação \textit{background} mantenha
sua execução até que a aplicação \textit{foreground} termine. Para o segundo e
terceiro experimento cada aplicação é executada de modo que seja tanto
\textit{background} quanto \textit{foreground}, sendo construída dessa forma
uma matriz \textit{n x n} com todos os possíveis resultados.

A fim de observar o quanto o desempenho é afetado pela interferência gerada por
uma aplicação executada em outra máquina virtual, é feita a medida da
degradação a partir do desempenho padrão de uma aplicação, essa medida
denomina-se pontuação normalizada. Assim, para calcular a pontuação normalizada
de uma aplicação, primeiro é definida a pontuação de desempenho inativa que é a
pontuação de uma aplicação quando executada contra uma máquina virtual inativa,
ou seja, sem nenhuma aplicação executando. Então, em seguida é feito o cálculo
da pontuação normalizada de uma aplicação F contra B, dividindo a pontuação de
desempenho de F contra B pela pontuação de desempenho inativa de F. Assim
define-se NS(F@B), como sendo a pontuação normalizada de F contra B, sendo
PD(F@B) a pontuação de desempenho de uma aplicação F contra uma aplicação B:

\begin{equation}
\label{eq:degradation}
NS(F@B) = PD(F@B)/PD(F@Inativo)
\end{equation}

A partir disso, é feito cálculo do desempenho combinado de duas aplicações, F e
B, em cada máquina virtual:

\begin{equation}
\label{eq:combined}
NS ( F + B ) = NS ( F @ B ) + NS ( B @ F )
\end{equation}


\begin{table}[!htb]
\centering
\caption{Aplicações utilizadas para geração de cargas e trabalho.}
\label{table-aplications}
\resizebox{0.45\textwidth}{!}{
\begin{tabular}{|l|c|c|}
\hline
Nome        & \multicolumn{1}{l|}{Maior Recurso Utilizado} & \multicolumn{1}{l|}{Medida de Desempenho} \\ \hline
Add\_double & CPU                                          & Pontuação                                  \\ \hline
Bzip2       & Misto                                        & Tempo                                      \\ \hline
Cat         & Disco                                        & Tempo                                      \\ \hline
Cachebench  & Memória                                      & Pontuação                                  \\ \hline
Ccrypt      & Misto                                        & Tempo                                      \\ \hline
Cp          & Disco                                        & Tempo                                      \\ \hline
Dd          & Disco                                        & Tempo                                      \\ \hline
Grep        & Disco                                        & Tempo                                      \\ \hline
Gzip        & Misto                                        & Tempo                                      \\ \hline
Iozone      & Disco                                        & Pontuação                                  \\ \hline
Make        & Misto                                        & Tempo                                      \\ \hline
Povray      & Misto                                        & Tempo                                      \\ \hline
\end{tabular}}
\end{table}

Sendo NS(F@B) e  NS(B@F) medidos em dois testes separados. Para medida de
desempenho, são utilizadas as pontuações geradas pelas próprias aplicações.
Entretanto, algumas aplicações, aquelas que não são voltadas para
\textit{benchmark}, não geram pontuações explícitas. Para essas aplicações,
fora definido como o inverso do tempo necessário para sua execuação, como
pontuação de desempenho. Na Tabela \ref{table-aplications} é apresentado o
maior recurso utilizado bem como a medida de desempenho utilizada para cada
ferramenta. Dessa forma, uma pontuação normalizada(F@B) próxima ou igual a 1 é
um resultado que indica que o desempenho de uma aplicação \textit{F} contra
\textit{B} sofreu baixa degradação.

