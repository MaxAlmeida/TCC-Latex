\chapter{Introdução}
\addcontentsline{toc}{chapter}{Introdução}
\section{Contextualizaç\~ao}
Ultimamente, devido a tendências como computação em nuvem, TI verde e a consolidação de servidores, a virtualização vem ganhando cada vez mais importância.Antigamente usada para uso mais eficiente de recursos físicos de mainframes, hoje em dia a virtualização é novamente utilizada para executar múltiplas máquinas virtuais em uma infraestrutura compartilhada, aumentando dessa forma a utilização de recursos, promovendo flexibilidade e centralizando a administração[huber].Ainda segundo popiolek, essa popularidade se deve também ao amplo uso de infraestruturas distribuidas por parte de sistemas computacionais modernos, o que colabora para o desenvolvimento de aplicações colaborativas e promove o compartilhamento de recursos remotos.

Neste contexto, a computação em nuvem, alinhada à virtualização, permite um conjunto de servidores físicos disponibilizar dezenas ou centenas de máquinas virtuais.Desse modo, proporciona-se aumento na escalabilidade, maximinizando o uso de recursos[Popiolek]. Entretanto, antes de migrar apliações de ambientes não virtuais para ambientes virtuais é necessário entender como será o desempenho dessas apliações nesse novo ambiente[fabricio benevuto].A adoção de servidores virtualizados vem com o aumento de custo na complexidade e dinâmica do sistema.O aumento da dinâmica é causada pela falta de controle direto sobre o hardware, pelas interações complexas entre as aplicações e cargas de trabalho que compartilham os mesmos recursos físicos introduzindo novos desafios em sistemas gestão[huber].

De acordo com Kooh, observa-se que as tecnologias atuais que possibilitam a criação de máquinas virtuais não fornece uma efetiva isolação de performance.Enquanto o hypervisor( monitor de máquinas virtuais), reparte os recursos e os aloca para as VMs, o comportamento de cada VM ainda pode afetar o desempenho das outras de maneira negativa, devido ao uso de recursos compartilhados no sistema.  

Alem disso, o custo de desempenho de virtualização pode variar de forma significativa dependendo da configuração do ambiente virtual.Uma escolha de configuração que afeta criticamente o desempenho do sistema é a alocação de CPUs a várias máquinas virtuais.[fabricio benevuto].

A partir do que foi abodardado até agora,entende-se que é fundamental compreender os fatores que impactam no desempenho de ambientes virtualizados, bem como quantificar e avaliar o desempenho de aplicações em tais ambientes para que possa implanta-las e configura-las adequadamente.Tendo isso em vista, torna-se importante o conhecimento de ferramentas e técnicas ou metodologias que auxiliem nas atividades relacionadas a análise de desempenho em ambientes virtualizados. 

\section{Quest\~ao de pesquisa}
De acordo com o contexto apresentado, este trabalho visa responder a seguinte questão: É possível definir uma metodologia que possibilite identificar os fatores que influenciam no desempenho de ambiente virtualizados? 

\section{Justificativa}
A virtualização introduz um outro nivel de complexidade para a modelagem de servidores e analise de performances.Como datacenteres rapidamente adotaram a virtualização como meio principal para consolidar múltiplas aplicações em um servidor, tournou-se essencial que a performance de máquinas virtuais seja bem compreendidas[omesh ticko]. Para [huber] os provedores de plafaformas virtualizadas se deparam com as seguintes questões:

\begin{itemize}
  \item Qual performance teria um novo serviço disponibilizado em um infraestrutura virtualizado e quanto de recurso deve ser alocado para tal serviço?

  \item Quanto de configuração deve ser adaptado para evitar o surgimento de problemas decorrentes de variações repentinas de cargas de trabalho? 
\end{itemize}

\section{Objetivo Geral}
Desenvolver uma metodologia que promova, a partir técnicas e ferramentas, a identificação de fatores que influenciam no desempenho de ambiente virtualizados, auxiliando desse modo a migração de aplicações para servidores virtualizados bem como no compreendimento de quais são os limites de uso de recursos físicos de um servidor sem ter a perda de desempenho.   
\section{Objetivos específicos}

\section{Estrutura do trabalho}

