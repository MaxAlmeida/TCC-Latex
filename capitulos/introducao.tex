arising\chapter{Introdução}
\addcontentsline{toc}{chapter}{Introdução}
\section{Contextualizaç\~ao}
Ultimamente, devido a tendências como computação em nuvem, TI verde e a consolidação de servidores, a virtualização vem ganhando cada vez mais importância. Antigamente usada para uso mais eficiente de recursos físicos de mainframes, hoje em dia a virtualização é novamente utilizada para executar múltiplas máquinas virtuais em uma infraestrutura compartilhada, aumentando dessa forma a utilização de recursos, promovendo flexibilidade e centralizando a administração\cite{huber2011}. Ainda segundo popiolek, essa popularidade se deve também ao amplo uso de infraestruturas distribuidas por parte de sistemas computacionais modernos, o que colabora para o desenvolvimento de aplicações colaborativas e promove o compartilhamento de recursos remotos.

Neste contexto, a computação em nuvem, alinhada à virtualização, permite um conjunto de servidores físicos disponibilizar dezenas ou centenas de máquinas virtuais.Desse modo, proporciona-se aumento na escalabilidade, maximinizando o uso de recursos[Popiolek]. Entretanto, antes de migrar apliações de ambientes não virtuais para ambientes virtuais é necessário entender como será o desempenho dessas apliações nesse novo ambiente[fabricio benevuto]. A adoção de servidores virtualizados vem com o aumento de custo na complexidade e dinâmica do sistema.O aumento da dinâmica é causada pela falta de controle direto sobre o hardware, pelas interações complexas entre as aplicações e cargas de trabalho que compartilham os mesmos recursos físicos introduzindo novos desafios em sistemas gestão[huber].

De acordo com Kooh, observa-se que as tecnologias atuais que possibilitam a criação de máquinas virtuais não fornece uma efetiva isolação de performance.Enquanto o hypervisor( monitor de máquinas virtuais), reparte os recursos e os aloca para as VMs, o comportamento de cada VM ainda pode afetar o desempenho das outras de maneira negativa, devido ao uso de recursos compartilhados no sistema.  

Além disso, o custo de desempenho de virtualização pode variar de forma significativa dependendo da configuração do ambiente virtual.Uma escolha de configuração que afeta criticamente o desempenho do sistema é a alocação de CPUs a várias máquinas virtuais.[fabricio benevuto].

A partir do que foi abodardado até agora, entende-se que é fundamental compreender os fatores que impactam no desempenho de ambientes virtualizados, bem como quantificar e avaliar o desempenho de aplicações em tais ambientes para que possa implanta-las e configura-las adequadamente. Tendo isso em vista, torna-se importante o conhecimento de ferramentas e técnicas ou metodologias que auxiliem nas atividades relacionadas a análise de desempenho em ambientes virtualizados. 

\section{Quest\~ao de pesquisa}
De acordo com o contexto apresentado, este trabalho visa responder a seguinte questão: É possível definir uma metodologia que possibilite identificar os fatores que influenciam no desempenho de ambiente virtualizados? 

\section{Justificativa}
A virtualização introduz um outro nível de complexidade para a modelagem de servidores e análise de performances. Como CPD's rapidamente adotaram a virtualização como meio principal para consolidar múltiplas aplicações em um servidor, tornou-se essencial que a performance de máquinas virtuais seja bem compreendidas[omesh ticko]. Para [huber] os provedores de plafaformas virtualizadas se deparam com as seguintes questões:

\begin{itemize}
  \item Qual performance teria um novo serviço disponibilizado em um infraestrutura virtualizado e quanto de recurso deve ser alocado para tal serviço?

  \item Como a configuração do sistema deve ser adaptada para evitar problemas de performance decorrentes de mudanças de cargas de trabalho?
\end{itemize}


Tais questões evidenciam a necessidade existente de se analisar os fatores que colaboram para a perda de desempenho em um ambiente virtualizado, bem como determinar quais configurações seriam adequadas para que tais fatores sejam minimizados sem que haja também o desperdício de recursos de hardware. Desse modo, em um cenário hipotético, sem ter um estudo ou uma metodologia prévia referente à análise de desempenho em ambientes virtualizados,e dado o uso dos mesmos no contexto de computação em nuvem, um profissional na área de computação em nuvem ao se deparar com problemas relacionados a desempenho, pode tomar decisôes equivocadas ocasionando perda de tempo na disponibilização de serviços em nuvem.

\section{Objetivo Geral}
Desenvolver uma metodologia que promova, a partir de técnicas e ferramentas, a identificação de fatores que influenciam no desempenho de ambiente virtualizados, auxiliando desse modo a migração de aplicações para servidores virtualizados bem como no compreendimento de quais são os limites de uso de recursos físicos de um servidor sem ter a perda de desempenho.   
\section{Objetivos específicos}

\begin{itemize}
\item Realizar estudo bibliográfico sobre virtualização para melhor entendimento dos conceitos
\item Efetuar estudo prévio de trabalhos ja realizados relacionados á análise de desempenho em ambientes virtualizados.
\item Elicitar principais suportes tecnológicos no que diz respeito a geração de cargas de trabalho e avaliação de métricas voltados para desempenho de sistemas virtualizados.
\item Definir cenários envolvendo diferentes configurações (e.g. número de CPU's, quantidade de máquinas virtuais em execução) para testes da metodologia proposta.
\item Aplicar metodologia proposta em uma infraestrturua de nuvem privada e apresentar a análise de resultados.
\end{itemize}

\section{Estrutura do trabalho}

Este trabalho está organizado da seguinte forma:
\begin{itemize}
\item Referencial Teórico - Neste capítulo, são apresentadas definições e conceitos relacionados a virtualização.
\item Suporte Tecnológico - São descritas as ferramentas utilizadas para o desenvolvimento deste trabalho 
\item Proposta - Descreve a proposta a partir de trabalhos relacionados, detalhando-se os procedimentos e métodos que irão compor a metodologia proposta bem como quais cenários serão utilizados para análise
\item Metodologia - Apresenta o fluxograma das atividades que serão realizadas bem como um cronograma deste conclusão deste trabalho.
\item Considerações finais - Apresenta uma breve reflexão sobre o trabalho proposto.

\end{itemize}

