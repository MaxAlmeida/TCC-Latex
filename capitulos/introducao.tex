\chapter{Introdução}
\addcontentsline{toc}{chapter}{Introdução}
\section{Contextualizaç\~ao}

Ultimamente, devido a tendências como computação em nuvem, TI verde e a consolidação de servidores, a virtualizaçao vem ganhando cada vez mais importância.Antigamente usada para uso mais eficiente de recursos físicos de mainframes, hoje em dia a virtualização é novamente utilizada para executar múltiplos servidores virtuais em uma infraestrutura compartilhada, aumentando dessa forma a utilização de recursos, promovendo flexibilidade e centralizando a administração[huber]. Ainda segundo [ fulano}   

\section{Quest\~ao de pesquisa}
De acordo com o contexto apresentado, este trabalho visa responder a seguinte questão: É possível definir uma metodologia que possibilite identificar os fatores que influenciam no desempenho de ambiente virtualizados? 

\section{Justificativa}

\section{Objetivo Geral}
Desenvolver uma metodologia que promova, a partir técnicas e ferramentas, a identificação de fatores que influenciam no desempenho de ambiente virtualizados, auxiliando desse modo a migração de aplicações para servidores virtualizados bem como no compreendimento de quais são os limites de uso de recursos físicos de um servidor sem ter a perda de desempenho.   
\section{Objetivos específicos}

\section{Estrutura do trabalho}

