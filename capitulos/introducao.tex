\chapter{Introdução}
\addcontentsline{toc}{chapter}{Introdução}
\section{Contextualizaç\~ao}

As pessoas estão inseridas na sociedade por meio das relações que desenvolvem durante toda sua vida. Primeiramente, essas relações ocorrem no âmbito familiar, em seguida na escola, na comunidade em que vivem e no trabalho. As relações que as pessoas desenvolvem e mantêm permitem fortalecer a esfera social. A própria natureza humana nos liga a outras pessoas, estruturando a sociedade em rede (TOMAÉL; ALCARÁ; DI CHIARA, 2005). Dessa maneira, quando se investiga o conceito de rede social, surge a ideia de um conjunto de nós interligados entre si, como uma teia que ocupa um determinado espaço em um ambiente (NEWMAN, 2010). Os nós ou pontos estão ligados em pares e podem representar várias situações em áreas científicas e de interesses comuns (GOLLNER et al., 2014).

Em outras palavras, uma rede social é um grafo onde pessoas ou organizações (dependendo da aplicação) são representadas por nós conectados por arestas. Essas últimas podem corresponder tanto a fortes relacionamentos sociais quanto ao compartilhamento de alguma característica (FREITAS et al.,2008). Com base em seu dinamismo, as redes, dentro do ambiente organizacional, funcionam como espaços para o compartilhamento de informação e do conhecimento. Espaços que podem ser tanto presenciais quanto virtuais, em que pessoas com os mesmos objetivos trocam experiências, criando bases e gerando informações relevantes para o setor em que atuam (TOMAÉL; ALCARÁ; DI CHIARA, 2005).

Nesse contexto, vale destacar que segundo Miranda (MIRANDA,1999) o conhecimento pode ser classificado da seguinte maneira:
	\begin{itemize}
	\item Conhecimento tácito:  acúmulo de saber prático sobre um determinado assunto, que agrega 			     convicções, crenças, sentimentos, emoções e outros fatores ligados à experiência e à personalidade        	 de quem o detém;
	\end{itemize}

\section{Objetivos}

O objetivo principal deste trabalho é o monitoramento de métricas de código fonte na API do sistema operacional Android, essencialmente métricas orientadas a objetos, e fazer um estudo da evolução de seus valores nas diferentes versões da API, estudar as semelhanças com aplicativos do sistema e então verificar a possibilidade de utilizar os dados obtidos para auxiliar no desenvolvimento de aplicativos. Essencialmente, será realizada uma análise exploratória de métricas de código fonte na API do sistema Android e de aplicativos. Será então apresentada uma proposta de cálculo de similaridade entre aplicativos e a API Android, utilizando como exemplo de uso desse cálculo um aplicativo parcial desenvolvido sem monitoramento de métricas.

Objetivos específicos:
\begin{itemize}
\item Estudo da evolução de métricas de código fonte em diversas versões da API;
\item Estudo da correlação da API Android e de seus aplicativos;
\item Definição e análise de intervalos de referência para valores de métricas;
\item Proposta de utilização dos intervalos definidos para auxílio no desenvolvimento de aplicativos;
\end{itemize}

\section{Estrutura do trabalho}

Este documento está dividido em 4 capítulos. No Capítulo~\ref{cap:metodologia} são melhor explicados os objetivos do trabalho e os métodos para o alcance dos objetivos propostos. Em seguida, o Capítulo~\ref{cap:analise_exploratoria} apresenta discussões sobre análise dos dados coletados, validação dos mesmos, e uma proposta de aplicação dos valores definidos. O Capítulo~\ref{cap:elastic} apresenta um aplicativo Android parcial com base em um estudo de caso específico, desenvolvido neste trabalho com o objetivo de alcançar uma boa arquitetura a partir de padrões de projeto e sem o auxílio de métricas, para base de comparação. Por fim, são apresentadas as considerações finais sobre a análise dos dados coletados e verificação de similaridade, bem como sugestões para continuidade deste estudo.