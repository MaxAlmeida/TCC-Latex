\chapter{Introdução}
\addcontentsline{toc}{chapter}{Introdução}
\section{Contextualizaç\~ao}
Ultimamente, devido a tendências como computação em nuvem, TI verde e a consolidação de servidores, a virtualização vem ganhando cada vez mais importância. Antigamente usada para uso mais eficiente de recursos físicos de \textit{mainframes}, hoje em dia a virtualização é novamente utilizada para executar múltiplas máquinas virtuais em uma infraestrutura compartilhada, aumentando dessa forma a utilização de recursos, promovendo flexibilidade e centralizando a administração\cite{huber2011}. Segundo \citeonline{popiolek2012}, essa popularidade se deve também ao amplo uso de infraestruturas distribuidas por parte de sistemas computacionais modernos, o que colabora para o desenvolvimento de aplicações colaborativas e promove o compartilhamento de recursos remotos.

Neste contexto, a computação em nuvem, alinhada à virtualização, permite um conjunto de servidores físicos disponibilizar dezenas ou centenas de máquinas virtuais. Desse modo, proporciona-se aumento na escalabilidade, maximizando o uso de recursos \cite{popiolek2012}. Entretanto, antes de migrar aplicações de ambientes não virtuais para ambientes virtuais, é necessário entender como será o desempenho dessas aplicações nesse novo ambiente \cite{benevuto2006}. A adoção de servidores virtualizados vem com o aumento de custo na complexidade e dinâmica do sistema. O aumento da dinâmica é causada pela falta de controle direto sobre o hardware, pelas interações complexas entre as aplicações e cargas de trabalho que compartilham os mesmos recursos físicos, introduzindo novos desafios em sistemas gestão\cite{huber2011}.

De acordo com \citeonline{koh2007}, observa-se que as tecnologias atuais que possibilitam a criação de máquinas virtuais não fornece um isolamento efetivo. Enquanto o \textit{hypervisor} ( software responsável pela disponibilização e gerenciamento de máquinas virtuais), reparte os recursos e os aloca para as VMs, o comportamento de cada VM ainda pode afetar o desempenho das outras de maneira negativa, devido ao uso de recursos compartilhados no sistema.  

Além disso, o custo de desempenho de virtualização pode variar de forma significativa, dependendo da configuração do ambiente virtual. Uma escolha de configuração que afeta criticamente o desempenho do sistema é a alocação de CPUs à várias máquinas virtuais\cite{benevuto2006}.

A partir do que foi abordado até agora, entende-se que é fundamental compreender os fatores que impactam no desempenho de ambientes virtualizados, bem como quantificar e avaliar o desempenho de aplicações em tais ambientes para que possa implanta-las e configura-las adequadamente. Tendo isso em vista, torna-se importante o conhecimento de ferramentas e técnicas ou metodologias que auxiliem nas atividades relacionadas a análise de desempenho em ambientes virtualizados. 

%\section{Quest\~ao de pesquisa}
%De acordo com o contexto apresentado, este trabalho visa responder a seguinte questão: O tipo de virtualização e o tipo de monitor de máquinas virtuais influenciam na interferência de desempenho entre máquinas virtuais? 


\section{Justificativa}
A virtualização introduz um outro nível de complexidade para a modelagem de servidores e análise de performances. Como CPD's rapidamente adotaram a virtualização como meio principal para consolidar múltiplas aplicações em um servidor, torna-se essencial que a performance de máquinas virtuais seja bem compreendidas\cite{ticko2010}. Para \citeonline{huber2011} os provedores de plafaformas virtualizadas se deparam com as seguintes questões:

\begin{itemize}
  \item Qual performance teria um novo serviço disponibilizado em um infraestrutura virtualizado e quanto de recurso deve ser alocado para tal serviço?

  \item Como a configuração do sistema deve ser adaptada para evitar problemas de performance decorrentes de mudanças de cargas de trabalho?
\end{itemize}


Tais questões evidenciam a necessidade de analisar os fatores que colaboram para a perda de desempenho em um ambiente virtualizado, bem como determinar quais configurações seriam adequadas para que tais fatores sejam minimizados sem que haja também o desperdício de recursos de hardware. Desse modo, em um cenário hipotético, sem ter um estudo ou uma metodologia prévia referente à análise de desempenho em ambientes virtualizados, e dado o uso dos mesmos no contexto de computação em nuvem, um profissional na área de computação em nuvem ao se deparar com problemas relacionados a desempenho, pode tomar decisôes equivocadas ocasionando perda de tempo na disponibilização de serviços em nuvem.



\section{Objetivos}
%Desenvolver uma metodologia que promova, a partir de técnicas e ferramentas, a identificação de fatores que influenciam no desempenho de ambiente virtualizados, auxiliando desse modo a migração de aplicações para servidores virtualizados bem como no compreendimento de quais são os limites de uso de recursos físicos de um servidor sem ter a perda de desempenho.   

%Aplicar, a partir de trabalhos já efetuados, um estudo de interferência entre máquinas virtuais em seus desempenhos, de modo que seja avaliada a influência do tipo de virtualização e da ferramenta utilizada como monitor de máquinas virtuais, na queda de desempenho em máquinas virtuais, causada por essa interferência. 

A partir do contexto e da justificativa apresentada, o objetivo deste trabalho é aplicar um estudo de interferência de desempenho entre máquinas virtuais utilizando como estudo de caso a infraestrutura do LAPPIS. Como colaboração ao LAPPIS e garantia de um provimento de máquinas virtuais de fácil gerenciamento para realização do estudo, pretende-se também o estabelecimento de uma platafroma em nuvem nessa infraestrutura. Para isso, os seguintes objetivos devem ser atingidos: 

\begin{itemize}
\item Realizar estudo bibliográfico sobre virtualização para melhor entendimento dos conceitos
\item Efetuar estudo prévio de trabalhos ja realizados relacionados á análise de desempenho em ambientes virtualizados.
\item Implantar uma Plataforma em Nuvem nos servidores do LAPPIS.
\item Automatizar parcialmente a infraestrutura virtual do LAPPIS ( plataforma em nuvem, serviços oferecidos)
\item Estabelecer um ambiente para estudo de interferência de desempenho entre ambientes virtuais.
\item Aplicar a análise de desempenho em um dos servidores físicos do LAPPIS.
\item Analisar e documentar os resultados obtidos a partir da analise de desempenho.
\item Sugerir uma nova alocação de máquinas virtuais da infraestrutura com base na análise de desempenho


%\item Elicitar principais suportes tecnológicos no que diz respeito a geração de cargas de trabalho e avaliação de métricas voltados para desempenho de sistemas virtualizados.
%\item Definir cenários envolvendo diferentes configurações (e.g. número de CPU's, quantidade de máquinas virtuais em execução) para testes da metodologia proposta.
%\item Aplicar metodologia proposta em uma infraestrturua de nuvem privada e apresentar a análise de resultados.
\end{itemize}

\section{Estrutura do trabalho}

Este trabalho está organizado da seguinte forma:
\begin{itemize}
\item Referencial Teórico - Neste capítulo, são apresentadas definições e conceitos relacionados a virtualização.
\item Suporte Tecnológico - São descritas as ferramentas utilizadas para o desenvolvimento deste trabalho 
%\item Proposta - Descreve a proposta a partir de trabalhos relacionados, detalhando-se os procedimentos e métodos que irão compor a metodologia proposta bem como quais cenários serão utilizados para análise
\item Resultado Preeliminares - Apresenta os resultados alcançados até o momento nesste trabalho.
\item Considerações finais - Apresenta uma breve reflexão sobre o trabalho proposto.

\end{itemize}

