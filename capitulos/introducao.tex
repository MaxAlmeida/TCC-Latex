\chapter{Introdução}
\addcontentsline{toc}{chapter}{Introdução}
\section{Contextualizaç\~ao}

As pessoas estão inseridas na sociedade por meio das relações que desenvolvem durante toda sua vida. Primeiramente, essas relações ocorrem no âmbito familiar, em seguida na escola, na comunidade em que vivem e no trabalho. As relações que as pessoas desenvolvem e mantêm permitem fortalecer a esfera social. A própria natureza humana nos liga a outras pessoas, estruturando a sociedade em rede (TOMAÉL; ALCARÁ; DI CHIARA, 2005). Dessa maneira, quando se investiga o conceito de rede social, surge a ideia de um conjunto de nós interligados entre si, como uma teia que ocupa um determinado espaço em um ambiente (NEWMAN, 2010). Os nós ou pontos estão ligados em pares e podem representar várias situações em áreas científicas e de interesses comuns (GOLLNER et al., 2014).

Em outras palavras, uma rede social é um grafo onde pessoas ou organizações (dependendo da aplicação) são representadas por nós conectados por arestas. Essas últimas podem corresponder tanto a fortes relacionamentos sociais quanto ao compartilhamento de alguma característica (FREITAS et al.,2008). Com base em seu dinamismo, as redes, dentro do ambiente organizacional, funcionam como espaços para o compartilhamento de informação e do conhecimento. Espaços que podem ser tanto presenciais quanto virtuais, em que pessoas com os mesmos objetivos trocam experiências, criando bases e gerando informações relevantes para o setor em que atuam (TOMAÉL; ALCARÁ; DI CHIARA, 2005).

Nesse contexto, vale destacar que segundo Miranda (MIRANDA,1999) o conhecimento pode ser classificado da seguinte maneira:
	\begin{itemize}
	\item Conhecimento tácito:  acúmulo de saber prático sobre um determinado assunto, que agrega 			     convicções, crenças, sentimentos, emoções e outros fatores ligados à experiência e à personalidade        	 de quem o detém;
	\end{itemize}

\section{Objetivos}



\section{Estrutura do trabalho}

