\chapter{Referencial Teórico}
\label{cap:referencial_teorico}
\section{Virtualização}
No contexto da computação, frequetemente uma alteração tecnológica torna alguma idéia obsoleta e ela desaparece rapidamente. Entretanto, outra mudança tecnológica poderia reavivá-la \cite{tanebaum}. Este é o caso da virtualização, dado que sua utilização teve oscilações ao longo do tempo. A principal motivação para virtualização nos começo dos anos 70 era aumentar o nivel de compartilhamento e utilização dos caros recursos computacionais tais como \textit{mainframes}\cite{Daniel}. Nos anos 80 com a queda dos custos de hardware as grandes corporações trocaram os grandes e despediosos \textit{mainframes} por coleções de microcomputadores, tendo assim a virtualização caído em desuso. Seu ressurgimento só viria acontecer, nos anos 90 dentro de um contexto ao qual tinha-se o crescimento de novos paradigmas computacionais, tais como cliente-servidor  e sistemas \textit{peer-to-peer}, aos quais em suas estruturas eram constituídas basicamente de máquinas clientes conectadas à vários servidores.
\section{Computação em Nuvem}