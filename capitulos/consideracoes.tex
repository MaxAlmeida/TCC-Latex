\chapter{Considerações Finais}
\label{cap:consideracoes_finais}


Neste trabalho foram apresentados conceitos referentes a virtualização e a computação em nuvem, bem como uma breve abordagem das ferramentas utilizadas para provimento dos mesmos e trabalhos relacionados a interferência de máquinas virtuais em seus desempenhos. E como já fora dito nesses tópicos, um dos conceitos chaves da computação em nuvem é o melhor aproveitamento no uso de recursos de servidores físicos, algo que não vinha sendo feito no LAPPIS. Outro assunto bastante recorrente no que diz respeito a provimento de plataformas em nuvem, são as possíveis perdas de desempenhos relacionadas com a virtualização. Apartir disso, este trabalho alcançou resultados que englobam uma colaboração para o LAPPIS e consequetemente para a FGA, e também o estabelecimento de uma infraestrutura que possibilitasse o estudo sobre a interferência de desempenho entre máquinas virtuais. A seguir são listados os resultados alcançados:  

\begin{itemize}
  \item Implantação de uma Plataforma em Nuvem nos servidores do LAPPIS.
  \item Automatização parcial da infraestrutura virtual ( plataforma em nuvem, serviços oferecidos)
  \item Estabelecimento de um ambiente para estudo de interferência de desempenho entre ambientes virtuais.
  %\item Disponibilização de serviços de utilidade para a FGA: \textit{Redmine}, \textit{Dotproject}, \textit{Moodle}, \textit{Boca}, ambientes de homologação para o portal da FGA.
\end{itemize}

A implantação de uma plataforma em nuvem, promoveu um melhor aproveitamento no uso de recursos dos servidores físicos do LAPPIS, garantido melhor gerenciabilidade e facilidade no provimento de ambientes virtuais, possibilitando assim a disponibilização de serviços de utilidade para FGA, tais como \textit{Redmine}, \textit{Dotproject}, \textit{Moodle}, \textit{Boca}. A automatização no provimento da plataforma em nuvem e dos serviços oferecidos, a partir do uso das ferramentas \textit{Chef} e \textit{Chake}, garante que os procedimentos para provimento dos mesmos sejam documentados, colaborando assim para o compartilhamento de conhecimento entre os colaboradores. A partir disso, com todos os serviços sobre uma plataforma em nuvem gerenciável é possível efetuar testes de interferência com mais comodidade e segurança, possibilitando inclusive o benefício na disponibilização desses serviços, melhor alocando-os entre os servidores disponíveis.
 %Por fim, como consequência disso, têm se o ambiente necessário para que sejam efetuados os estudos referentes a interferência de desempenho entre ambientes virtuais.

As próximas atividades deste trabalho serão concentradas na análise da intererência de desempenhos entre ambientes virtuais. Para tal, como apresentado  no capítulo \ref{cap:referencial_teorico}, já fora feita uma pesquisa prévia de trabalhos relacionados sobre o tema. Desse modo, na tabela \ref{tab:cronograma} é apresentado um cronograma com as atividades pretendidas para este trabalho.


\begin{table}[!ht]
\centering
\caption{Cronograma para as próximas atividades}
\label{tab:cronograma}
\resizebox{\textwidth}{!}{%
\begin{tabular}{|c|l|l|l|l|}
\hline
Atividades                                                                     & Março                  & Abril                  & Maio                   & Junho                  \\ \hline
Avaliar trabalhos relacionados à desempenho de ambientes virtuais              & \multicolumn{1}{c|}{X} & \multicolumn{1}{c|}{X} &                        &                        \\
Definir os procedimentos adotados para análise de desempenho                   &                        & \multicolumn{1}{c|}{X} &                        &                        \\
Aplicar os procedimentos de análise de desempenho em um dos servidores físicos &                        & \multicolumn{1}{c|}{X} & \multicolumn{1}{c|}{X} &                        \\
Coletar resultados da análise de desempenho                                    &                        & \multicolumn{1}{c|}{X} & \multicolumn{1}{c|}{X} &                        \\
Analisar resultados da análise de desempenho                                   &                        &                        & \multicolumn{1}{c|}{X} & \multicolumn{1}{c|}{}  \\ 
Documentar resultados da análise de desempenho                                 &                        &                        & \multicolumn{1}{c|}{X} & \multicolumn{1}{c|}{X} \\ 
Apresentar TCC à banca avaliadora                                              &                        &                        &                        & \multicolumn{1}{c|}{X} \\ \hline
\end{tabular}%
}
\end{table}

%A partir do referencial teórico foram apresentados os diversos conceitos que envolve virtualização, bem como uma abordagem geral sobre os modelos de computação em nuvem. Assim a apartir do exposto pelos trabalhos relacionados referentes a desempenho de máquinas virtuais em ambientes virtualizados esse trabalho tem como proposta aplicar um estudo de interferência de desempenho a partir de trabalhos já realizados. Para tal, seria necessário que fosse implementado uma infraestrutura de virtualização que possibilitasse tal estudo. Assim, como meio de colaborar com a infraestrutura do laboratório do LAPPIS, disponibilou-se uma infraestrutura em nuvem facilitasse a disponibilização de serviços necessárrios