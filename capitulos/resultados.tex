\chapter{Resultados Preeliminares}
\section{Estudo de Caso}
O LAPPIS possui uma infraestrutura de servidores que vinha possibilitando a disponibilização de serviços e ferramentas de utilidades para FGA. Destaca-se as ferramentas Redmine e Dotproject utilizadas nas disciplinas de metodologia de desenvolvimento de sofware e Gestão de Portiflólio e Produto, bem como possibilitava a disponibilização de máquinas virtuais utilizadas como ambientes de testes para desenvolvimento do Portal do Software Público, e para sistemas que estavam sendo desenvolvidos também pelo  lappis, tais como o SRA(sistema de registro de atendimento) e o SGD(sistema de gestão de desempenho). Entretanto tais recursos físicos vinham sendo subutilizados, devido aos seguintes fatores.
\begin{itemize}
 \item Versão do hypervisor desatualizada.
 \item Ausência de uma interface de gestão para máquinas virtuais.
 \item centralização do conhecimento.
\end{itemize}
      
      O hypervisor utilizado para disponibilização máquinas virtuais era o XEN na versão 4.1. Desse modo, com  o hypervisor nessa versão era impossível a disponibilização de máquinas virtuais com a versão de sistemas operacionais mais recentes tais como Debian 7, Debian 8 e Centos 7. O que tornava difícil também, a tarefa de disponibilizar ambientes de testes, com sistemas operacionais atualizados, para sistemas em desenvolvimento pelo LAPPIS. A falta de uma interface de gerenciamento dificultava atividades triviais tais como instanciação, criação de snapshots e migração de máquinas virtuais bem como visibilidade de uso de recursos. Por fim, a centralização do conhecimento impactava uma dependência problemática do profissional responsável pela implementação dessa infraestrutura. Assim, na sua ausência a equipe por parte do lappis responsável por essa infraestrutura, encontrou sérias dificuldades em manter a disponibilização de ambientes virtuais. Essa baixa visibilidade dos procedimentos adotados na infraestrutura, também promovia insegurança por parte da equipe em arriscar no desenvolvimento de mudanças relacionadas à essa infraestrutura. Dessa modo, o de recursos de harwdare disponíveis para provimento de serviços úteis tanto para o LAPPIS quanto para a FGA estava comprometida. 
      
      Apartir disso, dado a inviabilidade de continuar com essa infrestrutura, chegou-se a conclusão que o melhor caminho a ser adotado era a reformulação da mesma.Desse modo,adotou-se os seguintes procedimentos:
\begin{itemize}
      \item	Investigação de soluções em nuvem que atendessem as necessidades do lappis
      \item Instalação de um hypervisor atualizado
      \item Migração de máquinas virtuais para um dos servidores, afim de deixar o outro "livre" para testes de soluções de nuvem e de hypervisor.
      \item Consolidação de toda infraestrutura física sob as novas soluções de nuvem e do hypervisor.
\end{itemize}      
      
      
Desse modo, com os próprios colaboradores do LAPPIS desenvolvendo esse tipo de inciativa, a expectativa era que o problema relacionado com a centralização do conhecimento fosse sanado. A implementação de uma solução de nuvem, e consequetemente, uso de outro hypervisor ou até mesmo o próprio XEN atualizado proporcionaria a solução dos problemas relacionados a falta de gerenciabilidade e a disponibilização de máquinas virtuais com sistemas operacionais atualizados, respectivamente.        
                
\section{Infraestrutura}
A infraestrutura basicamente é composta de três servidores físicos e de máquinas virtuais que compartilham o uso de recursos desses servidores. Dois desses possuem a mesmas configurações: blablablablalbal. Esses servidores são identificados como \textit{Solarian} e \textit{Imperius}. O terceiro servidor físico possui a seguinte configuração . 

