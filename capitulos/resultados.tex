\chapter{Resultados Preeliminares}
O Lappis possui uma infraestrutura de servidores que vinha possibilitando a disponibilização de serviços e ferramentas de utilidades para FGA. Destaca-se as ferramentas Redmine e Dotproject utilizadas nas disciplinas de metodologia de desenvolvimento de sofware e Gestão de Portiflólio e Produto, bem como possibilitava a disponibilização de ambientes virtuais utilizados como ambientes de testes para aplicações utilizadas para desenvolvimento do Portal do software Público, e para sistemas que estavam sendo desenvolvidos também pelo  lappis, tais como o SRA(sistema de registro de atendimento) e o SGD(sistema de gestão de desempenho). Entretanto tais recursos físicos vinham sendo subutilizados, devido as seguintes questões técnicas.
      -Versão do hypverisor desatualizada
      -Ausência de uma interface de gestão
      -Centralização de conhecimento.
      
\section{Infraestrutura}

