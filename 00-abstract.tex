\begin{resumo}
Com o recente crescimento da computação em nuvem, tornaram-se recorrentes
questionamentos feitos com relação a perda de desempenho em ambientes
virutalizados. Neste contexto, realizamos uma série de coleta de dados,
desenvolvemos procedimentos e selecionamos um conjunto de ferramentas
tipicamente utilizadas para medição de desempenho em sistemas computacionais.
Nossos resultados mostram que o grau de interferência é variável de acordo com
o tipo de aplicação que está sendo executada nas máquinas virtuais, e que é
possível aplicar modelos estatísticos para predição de desempenho de uma
aplicação.
\end{resumo}

\begin{abstract}
Cloud computing is growing and there are recurrent questions about performance
loss in virtualized environments. In this context, we have performed several
data collection, developed procedures and selected a set of tools typically
used to measure performance in computational systems. Our results show that the
interference degree is variable according to the type of application that is
running in the virtual machines. In addition, it is possible to apply
statistical models for performance prediction of an application.
\end{abstract}

\begin{IEEEkeywords}
Cloud Computing, Virtualization, Performance Systems.
\end{IEEEkeywords}
