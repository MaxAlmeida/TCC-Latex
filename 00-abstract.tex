\begin{resumo}
Com o recente crescimento da computação em nuvem, tornaram-se recorrentes questionamentos feitos com relação a perda de desempenho em ambientes virutalizados. Com isso, a partir de trabalhos já realizados, esse artigo propõe a aplicação de um estudo de interferência de desempenho em máquinas virtuais. Para coletar os dados, foram desenvolvidos procedimentos e selecionadas um conjunto de  ferramentas tipicamente utilizadas para medição de desempenho em sistemas computacionais. Os resultados deste trabalho mostram que o grau de interferência é variável de acordo com o tipo de aplicação que está sendo executada nas máquinas virtuais. E que, com os dados obtidos, é possível aplicar modelos estatísticos para predição de desempenho de uma aplicação.
\end{resumo}

\begin{abstract}
Cloud computing is growing and there are recurrent questions about performance loss in virtualized environments. Based on related works, this paper proposes the application of a performance interference study on virtual machines. To collect the data, we have developed procedures and selected a set of tools typically used to measure performance in computational systems. The results of this work show that the interference degree is variable according to the type of application that is running in the virtual machines. In addition, with the collected data, it is possible to apply statistical models for performance prediction of an application.
\end{abstract}

\begin{IEEEkeywords}
Cloud Computing; Virtualization; Performance Systems; KVM.
\end{IEEEkeywords}
