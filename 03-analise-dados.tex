\section{Análise de dados}
\label{sec:analise-dados}

No trabalho de Koh \cite{koh2007} são utilizadas métricas de desempenho a nível
de sistema (denominadas \textit{System-level Workload Characteristics}) de
forma a analisar melhor o grau de interferência em diversos aspectos de sistema
na máquina virtual. Segundo Koh \cite{koh2007}, o fato desse tipo de métrica de
desempenho ser independente de qualquer tipo de microarquitetura subjacente
garante que seja possível ser feitas comparações através dos diferentes tipos
de servidores físicos.

Neste trabalho, para o \textit{KVM} chegou-se a ferramentas como o
\textit{iperf-kvm} e \textit{kvm-stat}. Entretanto, as informações apresentadas
por essas ferramentas não eram claras e, para o caso do \textit{kvm-stat}, não
detalhava por máquina virtual e sim para  \textit{hypervisor} inteiro. Mesmo no
\textit{OpenNebula} as informações apresentadas consistiam em uso de espaço em
disco, memória utilizada e quantidade de \textit{CPU} alocado por máquina
virtual, não sendo essas informações relevantes para o estudo proposto. Assim,
a abordagem escolhida foi o uso de uma ferramentas típica para monitoramento de
desempenho: \textit{iostat} para operações de disco e o \textit{mpstat} para
\textit{CPU}. Assim gerando as seguintes métricas coletadas neste trabalho:

\begin{itemize}

\item \textbf{Porcentagem de uso do \textit{CPU}}(\textit{cpuutil}):
Porcentagem de utilização de \textit{CPU} durante a execução de uma aplicação.
É obtida através da aplicação \textit{mpstat}.

\item \textbf{Requisições de escrita e leitura de disco por segundo
}(\textit{writes\_issued, reads\_issued}) e \textbf{Tempo gasto para escrita e
leitura do disco}(\textit{time\_writing, time\_reading}): Quantidade de
requisições no disco e o tempo para operações de escrita e leitura são bons
indicadores de operações de entrada e saída. Esses valores são coletados
utilizando o \textit{iostat}.

\end{itemize}

A análise dos dados obtidos referentes às métricas de desempenho e às
pontuações normalizadas das aplicações foi baseada no uso de técnicas de
análise multivariada com o intuito de predizer o desempenho das aplicações. Em
suma, usamos a média ponderada com o auxílio da análise por componente
principal (PCA), Regressão Linear e Regressão Polinomial. Para realização da
análise por componente principal utilizamos a ferramenta \textit{Scilab}com o
auxílio da biblioteca \textit{FACT}. Já para a análise de regressão linear e
polinomial utilizamos uma ferramenta de análise statística chamada \textit{R}.

De maneira geral, escolhemos uma aplicação ``U'' a partir do conjunto de
aplicações definidos anteriormente. Em seguida, ao utilizar o restante das
aplicações como um conjunto de referência (B1, B2, ..., Bn), foi feita a
predição da pontuação normalizada de ``U'' contra as outras aplicações (i.e,
NS(U@B1), NS(U@B2), ... NS(U@Bn)), das aplicações contra ``U'' (i.e, NS(B1@U),
NS(B2@U), ... NS(Bn@U)) e de ``U'' contra ele mesmo (NS(U@U)). Assim, para cada
técnica aplicada, foi comparado a pontuação predita com a pontuação real
medida. Para avaliação desse comparativo são calculados a média, mediana e o
maior erro da predição obtido, onde o cálculo do erro da predição é feito pela
equação \ref{eq:predict_error}. Esse procedimento foi feito com todas as
aplicações definidas:

\begin{equation}
\label{eq:predict_error}
\textrm{Erro da predição} = \frac{|\textrm{Pontuação real - Pontuação predita}|}{\textrm{Pontuação real}}
\end{equation}

\subsection{Método da Média Ponderada}

O método da média ponderada é baseado na similaridade entre duas aplicações.
Dessa forma, para calcular a similaridade, primeiro é feito o cálculo da
distância dos vetores de métricas de desempenho de duas aplicações. Dada o
número de dimensões desse vetor (5 métricas de desempenho a nível de sistema),
foi aplicada uma análise de componente principal (PCA) de modo que se reduzisse
a dimensionalidade dos dados sem perda significativa.

Assim, embora as \textit{p} variáveis de um vetor de dados sejam necessárias
para reproduzir a total variabilidade de um sistema, frequetemente muito dessa
variabilidade pode ser representada por uma pequena quantidade \textit{k} de
componentes principais. Dessa forma, os \textit{k} componentes principais podem
substituir  as \textit{p} variáveis iniciais, reduzindo o conjunto de dados
necessários para a análise \cite{johnson1988}.

Uma vez que os dados coletados foram convertidos pela análise de componentes
principal, escolhemos os componentes que representam a maior variabilidade
desses dados. Com isso, através dos dados obtidos, selecionamos três
componentes que representam 94\% da variança total (cada componente representa
53,8\%, 24\%, 16,3\% da variança, respectivamente).

Para cálcular a predição da pontuação de U@Bn, adota-se o seguinte procedimento
apresentado no trabalho de Koh \cite{koh2007}. Primeiro, em cima dos dados
transferidos para PCA calculamos a distância euclidiana do ponto desejado,
U@Bn, para todos os resultados de \textit{benchmark} obtidos e então foram
escolhidos os N pontos de dados mais próximos e definidos como um conjunto dos
mais próximos. A similaridade entre o ponto desejado e um ponto dentro do
conjunto dos mais próximos é definida como o inverso da distância. Então, foi
calculado o peso de cada dado no conjunto dos mais próximos proporcional a
similaridade:

\begin{equation}
\label{eq:prediction} 
w_i = s_i / \sum\limits_{i=1}^{N}s_i
\end{equation}

Onde S\textsubscript{i} é a similaridade de uma aplicação i dentro do conjunto
dos mais próximos. Por fim, a predição da pontuação de U@Bn foi calculada com a
seguinte fórmula:

\begin{equation}
\label{eq:simi} 
NS(U@Bn) = \sum\limits_{i=1}^{N}w_i \cdot NS(i)
\end{equation}

\subsection{Análise de Regressão Linear e Polinomial}

Análise de regressão é uma técnica estatística que tem como intuito analisar a
relação entre uma única variável dependente (critério) e várias variáveis
independentes (preditoras), possibilitando assim, apartir dos valores
conhecidos das variáveis independentes X\textsubscript{1}, X\textsubscript{2},
..., X\textsubscript{n} a predição do valor da variável dependente Y
\cite{hair}. Uma forma clássica da análise de regressão é a regressão linear,
que por vez declara que a variável dependente Y é uma função linear das
variáveis independentes. Assim, sua formulação é feita da seguinte maneira
\cite{johnson1988}:

\begin{equation}
\label{eq:linear} 
 \overline{Y} = a_0 + a_1 \cdot X_1 + a_2 \cdot X_2 + ... + a_n \cdot X_n
\end{equation}

Dessa forma, a regressão linear tem como objetivo determinar os coeficientes
a\textsubscript{0}, a\textsubscript{1}, ... , a\textsubscript{n}, utilizando
para isso o método dos mínimos quadrados, de modo a minimizar o erro
|\textoverline{Y} - Y| \cite{koh2007}. Por mais que a regressão linear seja
amplamente utilizada, sua forma clássica pode ainda assim não ser suficiente
para explicar uma boa quantidade de dados, principalmente aqueles que apresenta
um padrão não linear \cite{pantula}. Desse modo, uma outra forma de análise de
regressão é a regressão polinomial que se modela a partir de uma função
polinomial, e tem como proposta desenvolver aproximações para relações
curvilíneas ou não lineares. Os polinômios são transformações de pontência de
uma variável independente que acrescentam uma componente não-linear para cada
pontência adicional da variável independente. A construção de seu modelo
procede da seguinte maneira \cite{hair}:

\begin{equation}
\label{eq:polinomio} 
 \overline{Y} = a_0 + a_1 \cdot X_1 + a_2 \cdot X^2 + a_3 \cdot X^3 + ... + a_n \cdot X^n
\end{equation}

\begin{table}[!htb]
\centering
\caption{Tabela de coeficientes para regressão linear}
\label{tab:coeficiente_linear}
\begin{tabular}{|l|c|}
\hline
\multicolumn{1}{|c|}{X}   & \multicolumn{1}{l|}{Coeficientes} \\ \hline
write\_issued & 3,44                             \\ \hline
read\_issued  & 1,29                             \\ \hline
write\_time   & -1,37                            \\ \hline
read\_time    & -3,91                            \\ \hline
cpu\_util     & 5,15                             \\ \hline
a\textsubscript{0}           & 4,99                             \\ \hline
\end{tabular}
\end{table}

\begin{table}[]
\centering
\caption{Tabela de coeficientes para regressão polinomial}
\label{tab:coeficiente_poli}
\begin{tabular}{|c|c|c|c|}
\hline
\multicolumn{1}{|l|}{Variável} & \multicolumn{1}{l|}{Coeficientes} & \multicolumn{1}{l|}{Variável} & \multicolumn{1}{l|}{Coeficientes} \\ \hline
x\textsubscript{1}                             & -6,26E-03                        & $x\textsubscript{1} \cdot x\textsubscript{4}$                         & 4,27E-05                         \\ \hline
x\textsubscript{1}\textsuperscript{2}                            & 1,26E-05                         & $x\textsubscript{2} \cdot x\textsubscript{4}$                       & -7,23E-05                        \\ \hline
x\textsubscript{2}                             & 1,49E-03                         & $x\textsubscript{3} \cdot x\textsubscript{4}$                                                & -4,00E-05                        \\ \hline
$x\textsubscript{1} \cdot x\textsubscript{2}$                          & 1,60E-05                         & x\textsubscript{4}\textsuperscript{2}                           & 3,13E-05                         \\ \hline
x\textsubscript{2}\textsuperscript{2}                            & 2,42E-06                         & x\textsubscript{5}                            & -1,17E-02                        \\ \hline
x\textsubscript{3}                             & 1,20E-02                         & $x\textsubscript{1} \cdot x\textsubscript{5}$                                                                       & 6,42E-05                         \\ \hline
$x\textsubscript{1} \cdot x\textsubscript{3}$                        & -2,39E-05                        & $x\textsubscript{2} \cdot x\textsubscript{5}$                                                                       & -1,60E-05                        \\ \hline
$x\textsubscript{2} \cdot x\textsubscript{3}$                          & -3,48E-05                        & $x\textsubscript{3} \cdot x\textsubscript{5}$                                                                       & -1,26E-04                        \\ \hline
x\textsubscript{3}\textsuperscript{2}                            & -5,93E-06                        &$ x\textsubscript{4} \cdot x\textsubscript{5}$                                                                       & 1,08E-04                         \\ \hline
x\textsubscript{4}                             & -9,41E-03                        & x\textsubscript{5}\textsuperscript{2}                         & 1,03E-04                         \\ \hline
\end{tabular}
\end{table}

As Tabelas \ref{tab:coeficiente_linear} e \ref{tab:coeficiente_poli} apresentam
os coeficentes encontrados para regressão linear e regresssão polinomial
respectivamente. Vale acrescentar que na análise de regressão um dos mecanismos
utilizadas para verificação da precisão de um modelo preditivo é o coeficiente
de derterminação (R\textsuperscript{2}). Seu valor varia de 0 a 1, indicando,
em termos de porcentagem, o quanto um modelo conseguie explicar os valores
observados. Assim, se o R\textsuperscript{2} é 0,70 isso significa que 70\% dos
dados coletados são explicados pelo modelo preditivo calculado. O cálculo de
R\textsuperscript{2} é motrada na equação \ref{eq:polinomio}:

\begin{equation}
\label{eq:polinomio}  
R^2 = \frac{ \sum\limits_{i=1}^{N}(\hat{y}_i - \overline{y})^2}{\sum\limits_{i=1}^{N}(y_i - \overline{y})^2}
\end{equation} 

onde \textoverline{y} é a média de todas as observações, y\textsubscript{i} é o
valor da observação individual i e \^{y}\textsubscript{i} é o valor previsto da
observação i \cite{hair}.

Neste trabalho aplicamos tanto a análise de regressão linear quanto a análise
de regressão polinomial, definindo como variável dependente a pontuação
normalizada e como variáveis independentes as métricas de desempenho a nível de
sistema. Os dados coletados, com os valores das métricas de desempenho a nível
de sistema e com as pontuações normalizadas, foram utilizados para a
determinação dos coeficientes. Uma vez que os coeficientes foram calculados
utilizando o método dos mínimos quadrados, aplica-se os dados das métricas de
desempenho a nível de sistema na equação com os coeficientes obtidos,
alcançando dessa forma a predição da pontuação normalizada. Por fim, também são
calculados os valores de R\textsuperscript{2}, a fim de verificar qual modelo
apresenta maior aplicabilidade dentro dos dados coletados.

