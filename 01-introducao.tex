\section{Introdução}
\label{sec:introducao}

Devido as tendências como computação em nuvem, TI verde e a consolidação de
servidores, a virtualização vem ganhando cada vez mais importância.
Anteriormente usada para o uso mais eficiente de recursos físicos de
\textit{mainframes}, hoje em dia, a virtualização é novamente utilizada para
executar múltiplas máquinas virtuais em uma infraestrutura compartilhada,
aumentando dessa forma a utilização de recursos, promovendo flexibilidade e
centralizando a administração \cite{huber2011}. Essa popularidade deve-se
também ao amplo uso de infraestruturas distribuídas por parte de sistemas
computacionais modernos, o que colabora para o desenvolvimento de aplicações
colaborativas e promove o compartilhamento de recursos remotos
\cite{popiolek2012}.

Neste contexto, a computação em nuvem, alinhada à virtualização, permite um
conjunto de servidores físicos disponibilizar dezenas ou centenas de máquinas
virtuais. Desse modo, proporciona-se aumento na escalabilidade, maximizando o
uso de recursos \cite{popiolek2012}. Entretanto, antes de migrar aplicações de
ambientes não virtuais para ambientes virtuais, é necessário entender como será
o desempenho dessas aplicações nesse novo ambiente \cite{benevuto2006}. A
adoção de servidores virtualizados vem com o aumento de custo na complexidade e
dinâmica do sistema. O aumento da dinâmica é causada pela falta de controle
direto sob o hardware, pelas interações complexas entre as aplicações e cargas
de trabalho que compartilham os mesmos recursos físicos. Em suma, o aumento da
complexidade é causada pela introdução de recursos virtuais que por sua vez
causa uma diferença entre a alocação de recursos físicos e lógicos
\cite{huber2011}.

De acordo com Koh \cite{koh2007}, observa-se que as tecnologias atuais que
possibilitam a criação de máquinas virtuais não fornecem um isolamento efetivo.
Enquanto o \textit{hypervisor} (software responsável pela disponibilização e
gerenciamento de máquinas virtuais) reparte os recursos e os aloca para as VMs,
o comportamento de cada VM ainda pode afetar o desempenho das outras de maneira
negativa devido ao uso de recursos compartilhados no sistema.

Nesse contexto, entende-se que é fundamental compreender os fatores que
impactam no desempenho de ambientes virtualizados, bem como quantificar e
avaliar o desempenho de aplicações em tais ambientes para que possa
implantá-las e configurá-las adequadamente. Tendo isso em vista, torna-se
importante o conhecimento de ferramentas e técnicas ou metodologias que
auxiliem nas atividades relacionadas à análise de desempenho em ambientes
virtualizados.

Assim, neste trabalho foi realizado um estudo de interferência de desempenho em
ambientes virtuais ocasionado por um conjunto de aplicações. Através das
métricas de desempenho de sistema foi possível observar que determinadas
aplicações, dado seu perfil de execução, ocasionam maior grau de interferência
do que outras quando executadas em máquinas virtuais que compartilham o mesmo
servidor. Além disso, a partir dos dados coletados, foi possível aplicar
mecanismos de predição de desempenho em função da carga de trabalho que está
sendo aplicada. Esses mecanismos de predição foram desenvolvidos por Koh
\cite{koh2007}. Adicionalmente, neste trabalho aplica-se uma análise de
regressão polinomial baseada nestes mecanismos. Um comparativo entre esses
mecanismos foi realizado, sendo possível discutir que a análise de regressão
polinomial foi capaz de predizer o desempenho com taxas de erros menores do que
os mecanismos apresentados no trabalho por Koh \cite{koh2007}.  

